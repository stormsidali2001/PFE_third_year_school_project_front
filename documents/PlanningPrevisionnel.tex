\documentclass[11pt,fleqn]{book} % Default font size and left-justified equations

\usepackage[top=3cm,bottom=3cm,left=3.2cm,right=3.2cm,headsep=10pt,letterpaper]{geometry} % Page margins

\usepackage{xcolor} % Required for specifying colors by name
\definecolor{ocre}{RGB}{52,177,201} % Define the orange color used for highlighting throughout the book
\usepackage{wallpaper}
\usepackage{mdframed}
\usepackage[top=2cm, bottom=2cm, outer=0cm, inner=0cm]{geometry}
% Font Settings
\usepackage{avant} % Use the Avantgarde font for headings
%\usepackage{times} % Use the Times font for headings
\usepackage{mathptmx} % Use the Adobe Times Roman as the default text font together with math symbols from the Sym­bol, Chancery and Computer Modern fonts
\graphicspath{ {figures/} }
\usepackage{array}
\usepackage[T1]{fontenc}
\usepackage{imakeidx}
\makeindex
\usepackage[totoc]{idxlayout}
\usepackage{tabularx}
\usepackage{caption}
\usepackage{microtype} % Slightly tweak font spacing for aesthetics
\usepackage[utf8]{inputenc} % Required for including letters with accents
\usepackage[T1]{fontenc} % Use 8-bit encoding that has 256 glyphs
\usepackage{hyperref}
% Bibliography
\usepackage[style=alphabetic,sorting=nyt,sortcites=true,autopunct=true,babel=hyphen,hyperref=true,abbreviate=false,backref=true,backend=biber]{biblatex}
\addbibresource{bibliography.bib} % BibTeX bibliography file
\defbibheading{bibempty}{}

\input{structure} % Insert the commands.tex file which contains the majority of the structure behind the template

\begin{document}
\title{Workflow de Validation}

%----------------------------------------------------------
%	TITLE PAGE
%----------------------------------------------------------
\begingroup
\ThisLRCornerWallPaper{1.0}{Pictures/cover page.png}
\endgroup

%----------------------------------------Version page ---------
\newpage
~\vfill
\thispagestyle{empty}

%-------------------	TABLE OF CONTENTS
\pagestyle{empty} % No headers
\tableofcontents
\pagestyle{fancy} % Print headers again

%---------------------------------
\chapter{Planning Previsionnel}
\section{Introduction}
Notre projet "Project101" passe par 6 phases :
\begin{itemize}
    \item Phase d’initialisation
    \item Phase d’analyse
    \item Phase de conception
    \item Phase de développement
    \item Phase de tests
    \item Phase de bilan
\end{itemize}

\section{Phase d’initialisation}
Dans cette phase qui dure une semaine De 06-03-2022 jusqu’à 13-03-202 la realisation de :
\begin{itemize}
    \item Les chartes : 06/03/2022 jusqu'à 09/03/2022
        \item Les fiches : 10/03/2022 jusqu'à 11/03/2022
    \item Les plans : après la réalisation de chartes et des fiches de 12/03/2022 jusqu'à 13/03/2022
\end{itemize}


\section{Phase d'analyse}
Dans cette phase qui dure une semaine De 15-03-2022 jusqu’à 21-03-2022 :
\begin{itemize}
    \item \textbf{Analyse des besoins :}De 15-03-2022 jusqu’à 19-03-2022 
    \item \textbf{Cahier de charges :}après l’analyse des besoins jusqu'à la fin de la phase
\end{itemize}


\section{Phase de conception}
Dans cette phase qui dure De 22-03-2022 jusqu’à 09-04-2022:
\begin{itemize}
    \item \textbf{L’étude analytique: } 22/03 jusqu'à 01/04
    \item \textbf{L’architecture de système: } 02/04 jusqu'à 05/04
    \item \textbf{La conception de la Base De Données: } 06/04 jusqu'à 09/04
\end{itemize}


\section{Phase de développement}
Cette phase dure un peut plus long De 10-04-2022 jusqu’à 30-05-2022
\begin{itemize}
    \item  L’implémentation de la Base De Données
    \item La création des interfaces
    \item La création des modules 
    \item  L’intégration des modules 
\end{itemize}


\section{Phase de tests}
Cette phase commence le 14/04 et dure jusqu'à 31/05 contient :
\begin{itemize}
    \item  Les tests unitaires
    \item Après les tests d’intégration
    \item  Et finalement les tests de validations
\end{itemize}


\section{Phase de Bilan}
C’est la dernière phase dure de 9-06-2022 jusqu’à 14-06-2022 dans laquelle il y a l’évaluation de projet.
\\
\\
\\
\textbf{En Plus}
\begin{itemize}
    \item \textbf{les ressources: } toute l’équipe (6 personnes)
    \item \textbf{La durée: } De 06-03-2022 jusqu'à 14/06 :100 j.
    \item \textbf{Les rôles: }
    \begin{itemize}
        \item  chef de projet (chef d’équipe)
        \item  développeur (tous)
        \item éditeur des docs (tous)
        \item testeur (membre)
        \item analyste (responsable de la qualité)
        \item designer web (tous)
    \end{itemize}
\end{itemize}

\newpage
\section{Diagramme de GANTT}
\begin{figure}[h]
    \centering
    \includegraphics[width=1\textwidth]{Pictures/gantt.PNG}
    \caption{Diagramme de GANTT}
    \label{fig:pca}
\end{figure}

\end{document}
