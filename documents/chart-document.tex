\documentclass[11pt,fleqn]{book} % Default font size and left-justified equations

\usepackage[top=3cm,bottom=3cm,left=3.2cm,right=3.2cm,headsep=10pt,letterpaper]{geometry} % Page margins

\usepackage{xcolor} % Required for specifying colors by name
\definecolor{ocre}{RGB}{52,177,201} % Define the orange color used for highlighting throughout the book

% Font Settings
\usepackage{avant} % Use the Avantgarde font for headings
%\usepackage{times} % Use the Times font for headings
\usepackage{mathptmx} % Use the Adobe Times Roman as the default text font together with math symbols from the Sym­bol, Chancery and Com­puter Modern fonts

\usepackage{makeidx}
\makeindex
\usepackage[totoc]{idxlayout}
\usepackage{tabularx}
\usepackage{microtype} % Slightly tweak font spacing for aesthetics
\usepackage[utf8]{inputenc} % Required for including letters with accents
\usepackage[T1]{fontenc} % Use 8-bit encoding that has 256 glyphs
\usepackage{hyperref}
% Bibliography
\usepackage[style=alphabetic,sorting=nyt,sortcites=true,autopunct=true,babel=hyphen,hyperref=true,abbreviate=false,backref=true,backend=biber]{biblatex}
\addbibresource{bibliography.bib} % BibTeX bibliography file
\defbibheading{bibempty}{}

\input{structure} % Insert the commands.tex file which contains the majority of the structure behind the template

\begin{document}
\title{chart des documents}
\begingroup
\thispagestyle{empty}
\centering
\vspace*{5cm}
\par\normalfont\fontsize{35}{35}\sffamily\selectfont
\textbf{charte des documents}\\
{\LARGE gestion des projets de fin d'etudes}\par % Book title
\vspace*{2cm}
\raggedright
{\Huge team members \\}
\vspace*{0.5cm}
{\huge 
Assoul sidali \\
Debza Houda\\
Touati Amel\\
Larouci Ghezala\\}\par % Author name
\endgroup
%----------------------------------------------------------------------------------------
%	TABLE OF CONTENTS
%----------------------------------------------------------------------------------------


\pagestyle{empty} % No headers

\tableofcontents % Print the table of contents itself

%\cleardoublepage % Forces the first chapter to start on an odd page so it's on the right

\pagestyle{fancy} % Print headers again
%----------------------chapter 01----------------%
\chapter{charte des documents}
\section{definition}
\section{objectif}
L'objectif principal d'une charte de document est de conserver et garder une cohérence 
entre l’ensemble des documents réaliser, donc ce guide garantie une identité visuelle 
homogène sur tous les supports de chaque phase.
\section{norme de page de garde}
La page de garde est importante car elle représente le premier élément visuel qu'un
lecteur / encadreur / consulteur va regarder avant de lire le document. Son aspect
visuel et ses éléments écrits sont importants à travailler pour donner une bonne
première impression. //
Cette page doit suivre un ensemble de normes cités ci-dessous :
\subsection{ En-tête de la page de garde}
Cette dernière contient :
\begin{itemize}
    \item  Le logo de l’entreprise ‘ItExperts’ à gauche.
    \item Le logo de l’école supérieur d’informatique de Sidi Bel Abbes au centre.
     \item Le logo du projet ‘Project101’ à droite.
\end{itemize}

\subsection{ Centre de la page de garde}
Elle doit contenir :
\begin{itemize}

   \item Titre du document.
    \item Thème du Projet 
    \item Date de rédaction du document 
\end{itemize}
\section{ La mise en forme}
\subsection{Police}
Tous les documents sont écrits avec "Tahoma" et "Verdana" et "Times New Roman" d'une maniere qui va etre detaille dans la section suivante
\subsection{Taille de l’écriture}

\begin{figure}[h]
    \centering
    \includegraphics[width=1\textwidth]{Pictures/couleurs.PNG}
    \label{fig:pca}
\end{figure}

\section{ Table des Références}
Une table de références dresse la liste des références d'un document légal, ainsi que les 
numéros des pages sur lesquelles les références apparaissent. Elle inclue les liens de la 
webographie et également des titres d’ouvrages.
\section{ Règles à suivre}
\begin{itemize}
    \item  Les fichiers doivent être nommé sous forme : Nom du projet suivi du nom du document avec un tiré du bas ‘-’ à la place des espaces. Exemple : ‘Project101-Charte-De-Document’.
    \item Tous les documents doivent être sous forme de pdf.
    \item La première lettre des différents champs du nom doit commencer par une majuscule et le reste en minuscule.
    \item Chaque phrase commence par une majuscule et se termine par un point.
    \item Le nom du fichier est alphanumérique
\end{itemize}

\end{document}
